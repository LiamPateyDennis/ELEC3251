\documentclass[12pt,twoside]{scrartcl}
\usepackage[a4paper,top=2cm,left=2cm,right=2cm,bottom=2cm,includefoot,includehead]{geometry}
\usepackage{graphicx}
\usepackage[numbers]{natbib} % Add this line for natbib package
\usepackage{fancyhdr}
\usepackage{hyperref}



\pagestyle{fancy} % Set the page style to fancy
\fancyhf{} % Clear all header and footer fields

% Define the header and footer content for odd and even pages
\fancyhead[CE,CO]{School of Electrical Engineering} % Left header on even pages, right header on odd pages
% \fancyhead[RE,LO]{Right Header} % Right header on even pages, left header on odd pages
\fancyfoot[LE,RO]{\thepage} % Page number on left footer of even pages and right footer of odd pages
\fancyfoot[RE,LO]{ELEC3251} % Left footer on odd pages, right footer on even pages

\begin{document}
\pagenumbering{arabic}
\setcounter{page}{1}
\begin{titlepage}
    \begin{center}

        \includegraphics[width=0.2\textwidth]{LOGO_Square.pdf}

        \vspace*{0.4cm}
        School of Electrical Engineering \\
        University of Newcastle
        
        \vspace{1cm}
        \huge
        \textbf{\textsf{ELEC3251 \\ Assignment 1}}

        \vspace{0.5cm}
        \large
        \textbf{\textsf{Practical and Theoretical Analysis of \\ Buck Converter and Flyback Converter}}

        \vspace{1.5cm}
        \normalsize
        \begin{tabular}{l|r}
            Liam Patey-Dennis & c3349900 \\
            Joshua Thomas & c3376353
        \end{tabular}
        \vfill    
    \end{center}
\end{titlepage}


\section{Buck Converter}
\subsection{Ideal Calculations}
The duty cycle required to achieve an output voltage of $V_{o} = 5$ V for an input supply voltage of $V_{d} = 12$ V can be determined using the DC transfer function of the Buck converter, see Equation \ref{equation:Buck_TF}. The resulting duty cycle required for the ideal circuit is $D \approx 41.67$\%.\par
\begin{equation}
\frac{V_o}{V_d} = D \label{equation:Buck_TF}
\end{equation}
The output voltage ripple ($\Delta V_o$) of an ideal Buck converter is given by Equation \ref{equation:Buck_ripple}. This equation can be rearranged, see Equation \ref{equation:Buck_cap}, to find the required capacitance for the low pass filter, provided that the duty cycle ($D$), filter inductance ($L$), and switching period ($T_{s}$) are known. For $T_{s} = 1/f_{s} = 10$ $\mu$s, $L = 1$ mH, $D = 41.67$\%, $\Delta V_{o} = 25$ mV, and $V_o = 5$ V, the required capacitance is $C = 1.458$ $\mu$F. \par
\begin{equation}
\frac{\Delta V_{o}}{V_{o}} = \frac{1}{8}\frac{T_{s}^{2}(1-D)}{LC} \label{equation:Buck_ripple}
\end{equation}
\begin{equation}
C = \frac{V_o}{\Delta V_{o}}\frac{T_{s}^{2}(1-D)}{8L} \label{equation:Buck_cap}
\end{equation}
The output power of the converter will be limited by the ratings of the components in the circuit. As the output voltage of the converter is required to remain constant at $5$ V, only the output current can be adjusted to suit the ratings of the components. The inductor selected for the circuit is the Murata \#1410516C, which has a maximum DC current of $1.6$ A \cite{RNX0}. A IRFZ24NPbF MOSFET has been selected for the switch, this component has a maximum DC current rating of $17$ A \cite{RNX1}. The selected diode is an SB120 which has a maximum DC current of $1.0$ A \cite{RNX2}. The inductor current will equal the output current, assuming the voltage across the capacitor remains constant. Therefore, the output current must be less than $1.6$ A, to avoid causing damage to the inductor. The diode will only conduct when the switch is off, therefore the DC current flowing through the diode is $I_{D} = (1-D)I_{o}$. For $I_{o} = 1.6$ A the diode current is $0.93$ A, which is less than the maximum rating of the device. Therefore, the load resistance must be selected such that $I_{o} \le 1.6$ A. Using Ohm’s law this inequality is equivalent to $R_{Load} \ge 3.125$ $\Omega$. The largest resistor provided in the laboratory kit is $3.9$ $\Omega$ so this resistance will be used for the load.\par
\vspace{5mm}
\noindent The continuous conduction mode (CCM) and discontinuous conduction mode (DCM) boundary occurs when the current flowing through the inductor reaches $0$ A. For the ideal Buck converter this will occur for a DC output current ($I_{oB}$) which can be found using Equation \ref{equation:Buck_DCM}. For the designed converter the minimum DC output current is $I_{oB} = 14.6$ mA, which is equivalent to an output load of $342$ $\Omega$.
\begin{equation}
I_{oB} \approx \frac{T_{s}V_{o}}{2L}(1-D) \label{equation:Buck_DCM}
\end{equation}
\pagebreak
\subsection{Ideal Simulations}

\newpage
\section{Flyback Converter}
\citep{BS412-EN}
\citep{jay1995write}
\newpage
\section{Conclusion}
This is blank text.
\newpage
\bibliographystyle{IEEEtranN}
\bibliography{references}

\end{document}